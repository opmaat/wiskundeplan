\documentclass{ximera}
\input{../preamble}
\addPrintStyle{..}
\begin{document}
    \author{Wiskunde Op Maat }
    \xmtitle{De vermenigvuldiging in polaire vorm}{}
     
    \label{xim:complexe_getallen_polair}  % goniometrisch ???
    \pdfOnly{\shorthandoff{"}}  % terrible hack to make [dutch]babel and tikz compatible

  

Het optellen van complexe getallen is erg eenvoudig in de cartesische schrijfwijze, het correspondeert immers met het optellen van vectoren. 
Bij de vermenigvuldigen van complexe getallen is de polaire schrijfwijze erg eenvoudig, ook deze bewerking correspondeert met een meetkundige operatie in het complexe vlak. 
Zij $z_1=r_1 (\cos \theta_1 + i \sin \theta_1 ) $ en $z_2 = r_2(\cos \theta_2 + i \sin \theta_2 )$ complexe getallen. Met behulp van de som- en verschilformules van goniometrische getallen levert een rechtstreekse berekening: 
\begin{align*}
    z_1\cdot z_2 &= r_1 (\cos \theta_1 + i \sin \theta_1 )\cdot r_2 (\cos\theta_2 + i \sin \theta_2 )\\
                 &= r_1 r_2 (\cos \theta_1 \cos \theta_2 - \sin \theta_1  \sin\theta_2 + i(\sin \theta_1 \cos \theta_2 + \cos \theta_1  \sin
    \theta_2 ))\\
                 &= r_1 r_2 ( \cos(\theta_1 + \theta_2) + i \sin(\theta_1 + \theta_2) \, ,
\end{align*}


De moduli $r_i$ worden vermenigvuldigd, en de argumenten $\theta_i$ worden opgeteld.
 
\begin{proposition}[Vermenigvuldiging van twee complexe getallen in goniometrische schrijfwijze]\label{eig:vermenigvuldiging_complexe_getallen}\nl
     
De \textit{modulus} van het product van twee complexe getallen is het \textit{product van de moduli} van die getallen.
 
Het \textit{argument} van het product van twee complexe getallen is de \textit{som van de argumenten} van die getallen.

Voor \( z_1 = r_1 (\cos \theta_1 + i \sin \theta_1 )\) en \(z_2 = r_2 (\cos\theta_2 + i \sin \theta_2 )\) wordt het product gegeven door  \[ z_1 \cdot z_2 = r_1 r_2 ( \cos(\theta_1 + \theta_2) + i \sin(\theta_1 + \theta_2) \]
\end{proposition}
 

\begin{exercise}
    Gegeven zijn de complexe getallen:
\begin{align*}
    z_1 &= 2 \left(\cos\frac{5\pi}{12} + i \sin\frac{5\pi}{12}\right), \\
    z_2 &= 3 \left(\cos\frac{5\pi}{6}+ i \sin\frac{5\pi}{6}\right), \\
\end{align*}
Bereken algebraïsch de volgende complexe getallen en geef het resultaat in cartesische en polaire vorm.

    \begin{question} $z_1 \cdot z_2$    \end{question}
    \begin{question} $(z_1)^2$          \end{question}
    \begin{question} $\frac{z_2}{z_1}$  \end{question}
  
\end{exercise}







\begin{xmuitweiding}[Vermenigvuldiging van 2 complexe getallen grafisch uitgelegd]
    % HIER MOET NOG EEN TIKZPICTURE BIJ 


We kunnen de vermenigvuldiging met een complex getal ook zien als een actie die we doen op punten in het vlak. Als we een bepaald complex getal $z = r(\cos\theta + i \sin \theta)$ fixeren, en eender welk complex getal vermenigvuldigen met die $z$, dan tellen we steeds $\theta$ bij het argument van het complex getal op, en vermenigvuldigen we de modulus van dat complex getal steeds met $r$.
 
% Als je focust op het effect van vermenigvuldiging met één complex getal, krijg je %%% vind ik precies wat een vreemde zin?
\begin{proposition}[Vermenigvuldiging met een complex getal in goniometrische schrijfwijze $z=r(\cos\theta + i\sin\theta)$]\nl
     
Vermenigvuldigen met $z = r(\cos\theta + i\sin\theta)$ betekent \textit{roteren} over een hoek $\theta$ en afstanden en lengtes \textit{herschalen} met een factor $r$.
\end{proposition}
 
Je kan experimenteren met deze voorstelling van de vermenigvuldiging door in onderstaande applet de complexe getallen $z_1$ en $z_2$ te verslepen, en dan via de vermenigvuldig-knop te zien hoe vermenigvuldigen met $z_2 = (r,\theta)$ inderdaad kan beschouwd worden als een \textit{rotatie} over een hoek $\theta$ gevolgd door een \textit{herschaling} met een factor $r$.
 
\geogebra{i0Lbx90s}{893}{451}

\begin{exercise}
    Teken devolgende complexe getallen \(z_1\) en \(z_2\) als punten van het complexe vlak. Bepaal zonder te rekenen waar haar het product \( z_1 z_2\) ligt. 
    \begin{question} \(z_1 = 1+1\) \(z_2 = i\)    \end{question}
    \begin{question} \(z_1 = 5\) \(z_2 = i - 1\)\end{question}
    \begin{question} \(z_1 = 2 + 2i\) \(z_2 = -2 - 2i \) \end{question}
\end{exercise}

\end{xmuitweiding}

% van gpt: 
% \begin{tikzpicture}
%     % Draw the axes
%     \draw[->] (-1, 0) -- (5, 0) node[right] {Re};
%     \draw[->] (0, -1) -- (0, 5) node[above] {Im};

%     % Define the complex numbers
%     \coordinate (z1) at (3, 1); % z1 = 3 + i
%     \coordinate (z2) at (2, 2); % z2 = 2 + 2i

%     % Plot z1
%     \draw[->, thick, blue] (0, 0) -- (z1) node[midway, above] {$z_1 = 3 + i$};

%     % Plot z2
%     \draw[->, thick, red] (0, 0) -- (z2) node[midway, left] {$z_2 = 2 + 2i$};

%     % Calculate z1 * z2 = (3 + i)(2 + 2i) = 6 + 6i + 2i - 2 = 4 + 8i
%     \coordinate (z1z2) at (4, 8);

%     % Plot z1 * z2
%     \draw[->, thick, purple] (0, 0) -- (z1z2) node[midway, right] {$z_1 \cdot z_2 = 4 + 8i$};

%     % Add grid
%     \draw[help lines] (-1, -1) grid (5, 9);

%     % Add labels
%     \foreach \x in {-1, 1, 2, 3, 4, 5}
%         \draw (\x, 1pt) -- (\x, -1pt) node[anchor=north] {\x};
%     \foreach \y in {-1, 1, 2, 3, 4, 5, 6, 7, 8, 9}
%         \draw (1pt, \y) -- (-1pt, \y) node[anchor=east] {\y};

%     % Mark angles
%     \draw pic["$\theta_1$", draw=blue, <->, angle eccentricity=1.2, angle radius=1cm] {angle=z1--(0,0)--(1,0)};
%     \draw pic["$\theta_2$", draw=red, <->, angle eccentricity=1.2, angle radius=0.8cm] {angle=z2--(0,0)--(1,0)};
%     \draw pic["$\theta_1 + \theta_2$", draw=purple, <->, angle eccentricity=1.2, angle radius=1.2cm] {angle=z1z2--(0,0)--(1,0)};
% \end{tikzpicture}



 
% \begin{xmuitweiding}[Inverse van een complex getal met poolcoördinaten]
% We zijn geïnteresseerd in het volgende probleem: stel dat je een complex getal $z$ hebt, met welk ander complex getal, zeg $w$, moet je $z$ vermenigvuldigen zodat het resultaat 1 geeft? Met andere woorden: hoe bepaal je de \textit{inverse} van een complex getal?
 
% Stel dat het complex getal $z$ zuiver reëel is, dus zodanig dat de cartesische schrijfwijze van $z$ van de vorm $z = a + 0i$ is. Dan is het makkelijk om te vinden wat de inverse is: door te vermenigvuldigen met $w = 1/a$ komen we op 1 uit. Het algemene geval is echter lastiger met cartesische coordinaten,
% % als $z = a + bi$, en $w = c + di$, dan wordt de vergelijking $z\cdot w = 1$:
% %$$
% %z\cdot w = (a+bi)\cdot (c+di) = (ac -  bd) + (bc + ad)i = 1 + 0i \, .
% %$$
% %Dit komt neer op een stelsel van twee vergelijkingen in twee onbekenden:
% %$$
% %\begin{cases}
% %ac - bd &= 1 \\
% %bc + ad &= 0
% %\end{cases}
% %$$
% maar zal makkelijker zijn met poolcoördinaten, aangezien het hier gaat over vermenigvuldigingen van complexe getallen. Als we de rekenregels voor het vermenigvuldigen van complexe getallen in goniometrische schrijfwijze in het achterhoofd houden, vinden we het volgende:
 
% \begin{proposition}
% Stel dat het complexe getal $z$ modulus $r$ en argument $\theta$ heeft. Dan is de inverse van $z$ het complexe getal met modulus $1/r$ en argument $-\theta$. 
% \end{proposition}
 
% Uiteraard bestaat de inverse dus niet voor het complex getal nul. Voor $z\neq 0$, met $(r, \theta)$ als poolcoördinaten, is de inverse inderdaad het complex getal $w$ met poolcoördinaten $\left(\frac{1}{r}, -\theta\right)$: de modulus van het product van complexe getallen is het product van de moduli, terwijl het argument van het product van complexe getallen de som van de argumenten is. Dus $z\cdot w$ heeft modulus 1, en argument 0: dit zijn precies de poolcoördinaten van het getal 1.
% \end{xmuitweiding}
 
%Indien de cartesiaanse vorm gegeven is, kan de modulus bepaald worden met de stelling van Pythagoras, en de hoek $\theta$ in de rechthoekige driehoek met rechthoekszijden $|a|$ en $|b|$. Merk op dat voor $a<0$, het complex getal in het tweede of derde kwadrant ligt. 
%Aangezien $\arctan$ per definitie steeds als resultaat een hoek in het interval $]-\pi/2,\pi/2[$ geeft, dient  in de uitdrukking voor het argument $\theta$ een term $\pi$ toegevoegd te worden om correct te zijn.
%Als $|a|$ of $|b|$ gelijk is aan 0, redeneer je niet in een driehoek, maar ligt het complex getal op \'e\'en van de assen en kan je het argument $\theta$ onmiddellijk aflezen op een figuur.
 
 
% verdere inhoud over complexe getallen uit Zomercursus 201x in kwadratische_vergelijking/veeltermen/...
 
%\begin{thebibliography}{9}
%\bibitem{nieuwe delta} P. Gevers, J. Anseeuw, J. De Langhe, G. Roels, H. Vercauter,
%\textit{Nieuwe Delta, 5/6 Complexe Getallen(6 - 8 uur)}, Leuven,
%Wolters Plantyn, 1994.
%\bibitem{Jennekens}E. Jennekens, G. Deen, \textit{Wiskunde '68, Wiskunde 5, Deel A, Matrices en
%complexe getallen}, Antwerpen, De Sikkel, 1972.
%\bibitem{Quaegebeur}
%J. Quaegebeur, \textit{Basisbegrippen en basistechnieken uit de
%wiskunde}, Leuven, Acco, 2004.
%\bibitem{Croft}A. Croft and R. Davison, \textit{Mathematics for engineers}, Pearson Education Limited, third edition, 2008.
%\end{thebibliography}
%
%\newpage
 
\end{document}