\documentclass{ximera}
\input{../preamble}
\addPrintStyle{..}
\begin{document}
    % Start specifieke settings:   
    \author{Zomercursus KU Leuven}
    \xmtitle{Machten van complexe getallen: formule van De Moivre}{}
    % Start inhoud ximera
     
    \label{xim:complexe_getallen_demoivre}
 

% LABEL AND AUTHOR MOET NOG WORDEN AANGEPAST 



De n-de macht van een complexe getal \(z^n\) is het n-voudig product \(z \cdot z \cdot z \cdot \dots \cdot z \). (n-keer invoegen). Met de vermenigvuldiging van complexe getallen kan dit n-voudig product uitgerekend worden. Ook hier zijn de berekeningen met de goniometrische schrijfwijze veel eenvoudiger. 

Het meest eenvoudige geval \(n=2\) is het kwadraat \(z^2\) van een complex getal. Dit is de vermenigvuldiging van het complex getal met zichzelf. Voor een complex getal \(z\) met als goniometrische schrijfwijze \(r(\cos \theta + i \sin \theta ) \) wordt dit: 

\[
z^2 = z \cdot z  = r(\cos \theta + i \sin \theta ) \cdot r(\cos \theta + i \sin \theta ) = r^2(\cos (\theta + \theta) + i \sin (\theta + \theta) ) =  r^2(\cos 2\theta + i \sin 2\theta ). 
\]

Het herhalen van deze methode levert dan hogere machten van een complex getal.  \(z^3 = z^2 \cdot z \). Het algemene resultaat staat bekend als de formule van DeMoivre: 
 
\begin{proposition}[Formule van De Moivre]\nl 
Voor een complex getal $z= r (\cos \theta + i \sin \theta )$ en elk natuurlijk getal $n$ geldt
\[
z^n = r^n(\cos n\theta + i \sin n\theta )
\]
\end{proposition}
 
\begin{example}
    We kunnen $(1+i)^8$ op 2 manieren berekenen:         
        \begin{enumerate}
            \item $(1+i)(1+i)=1+2i-1=2i$, dus $(1+i)(1+i)(1+i)(1+i)=(2i)^2=-4$, dus $(1+i)^8=(-4)^2=16$
            \item $1+i=\sqrt{2}(\cos \frac{\pi}{4}+i\sin\frac{\pi}{4})$, dus $(1+i)^8=(\sqrt{2})^8(\cos  \frac{8\pi}{4}+i\sin \frac{8\pi}{4})=2^4(\cos 2\pi +i \sin 2\pi)=16(1+i\cdot 0)=16$
        \end{enumerate}
         
    \end{example}
 
\begin{exercise} Bereken met de formule van De Moivre
    \begin{question} $(\sqrt 3 + i)^3 = \answer[onlineshowanswerbutton]{8i}$
    \end{question}
    \begin{question} $(-1-i)^{20} = \answer[onlineshowanswerbutton]{-1024}$
    \end{question}
    \begin{question} $(1+ i)^{21} = \answer[onlineshowanswerbutton]{-1024-1024i}$
    \end{question}
    \begin{question} $(-\sqrt 3 + i)^5 = \answer[onlineshowanswerbutton]{16\sqrt3+16i}$
    \end{question}
\end{exercise}


\begin{exercise}
    Gegeven is het complex getal \(z = \cos(\frac{3\pi}{4} + i\sin(\frac{3\pi}{4}))\). 
    \begin{question}
        Bereken \(z^0, z^1, z^2, z^3, \dots, z^8 \) en schrijf telkens het resultaat in cartesische vorm. 
    \end{question}
    \begin{question}
        Stel de opeenvolgende machten van \(z\) voor in het complexe vlak. Wat valt je op? 
    \end{question}
\end{exercise}






% NOG INVOEGEN OP MAAT; EEN MEER UITGEBREIDE  BESPREKEING VAN DE MEETKUNDIGE BETEKENIS ZIE KOEN; HIERVOOR IS GEOGEBRA EN KENNISCLIP OOK GESCHIKT. 


 
 
\end{document}