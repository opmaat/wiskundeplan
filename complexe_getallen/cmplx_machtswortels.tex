  \documentclass{ximera}
\input{../preamble}
\addPrintStyle{..}

\begin{document}
    \author{Wim Obbels}
    \xmtitle{Wortels van complexe getallen}{}     
    \label{xim:complexe_machtswortels}

    \providecommand{\fm}{\phantom{-}}
    \renewcommand{\fm}{\phantom{-}}

    Het begrip $n$-de machtswortel van de reële getallen kan op het eerste zicht ook eenvoudig worden gebruikt bij complexe getallen. 
    Er treden echter enkele nieuwe fenomenen op.

    \begin{definition}\nl

    Voor elk natuurlijk getal $n\in \Nnul$ en complex getal $w\in\C$ is een complex getal $z\in\C$ \textbf{een $n$-de machtswortel van $w$}  als de $n$-de macht van $z$ gelijk is aan $w$:
 
    \formulevb{z \text{ is }\textbf{een $n$-de machtswortel van $w$} \iff z^n = w}{2^4 = (-2)^4 = 16, \quad (-2i)^3 = 8i}
    % \formulevb{z \text{ is }\textbf{een $n$-de machtswortel van $w$} \iff z^n = w}{2^3 = 8, \quad (-1)^4 = i^4 = (-i)^4 = 1 ^4 = 1}

    \end{definition}



    \begin{example} Welke van volgende uitspraken zijn waar?
        \begin{question} 
            \choiceTrue  $\fm3$ is een derdermachtswortel van $27$
            \begin{feedback} want $3^3 = 27$\end{feedback}
        \end{question}
        \begin{question} 
            \choiceFalse  $-3$ is een derdermachtswortel van $27$
            \begin{feedback} want $(-3)^3 = -27 \neq 27 $\end{feedback}
        \end{question}
        \begin{question} 
            \choiceTrue  $i$ is een vierkantswortel van $-1$
            \begin{feedback} want $i^2= -1$\end{feedback}
        \end{question}
        \begin{question} 
            \choiceFalse  $2+3i$ is een vierkantswortel van $4+9i$
            \begin{feedback} want $(2+3i)^2 = 4 + 12i - 9 = -5+12i \neq 4+9i$\end{feedback}
        \end{question}
        \begin{question} 
            \choiceTrue  $2+3i$ is een vierkantswortel van $-5+12i$
            \begin{feedback} want $(2+3i)^2 = 4 + 12i - 9 = -5+12i$\end{feedback}
        \end{question}

    \end{example}

    \begin{remark}\nl

        \begin{itemize}
            \item Net zoals bij de reële getallen is het begrip $n$-de machtswortel van $a$ dus eigenlijk gewoon een naam voor \textit{een oplossing van de vergelijking $x^n = a$}.
            \item Bij de reële getallen was er een belangrijk onderscheid tussen $n$ even en $n$ oneven omdat negatieve getallen geen even machtswortels hebben, en positieve getallen twee even machtswortels hebben:
            \begin{center}
            \begin{tabular}{lll}
                $x^2 =\fm4$ & heeft twee reële oplossingen:     & $x=2$ \text{ en } $x=-2$ \\
                $x^2 =  -4$ & heeft geen reële oplossingen:     & $x^2 \neq -4, \forall x \in \R $\\
                $x^3 =\fm8$ & heeft een unieke reële oplossing: & $x=2$ \\
                $x^3 =  -8$ & heeft een unieke reële oplossing: & $x=-2$ \\
            \end{tabular}
            \end{center}
            Er zijn altijd ten hoogste twee reële $n$-de machtswortels, en als $a\in\R$ twee $n$-de machtswortel had  noemden we de unieke positieve \textbf{de} $n$-machtswortel, en noteerden die met $\sqrt[n]{a}$ of $a^{1/n}$. Voor $n$ even was dan ook $-\sqrt[n]{a}$ \textit{een} $n$-de machtswortel.
            \item Bij complexe getallen is de zaak enerzijds eenvoudiger: we zullen zien dat er \textit{altijd} precies $n$ $n$-de machtswortels zijn. Anderzijds is het niet meer mogelijk om over \textbf{de} $n$-de machtswortel te spreken. We zullen dan ook de notatie $\sqrt[n]{z}$ of $z^{1/n}$ niet gebruiken.
        \end{itemize}
        \end{remark}
    

% Bron: zomercursus KULeuven


Vergelijkingen van de vorm $z^n=a+bi$ met $z\in \C, n\in \Z$ zijn met de formule van De Moivre op te lossen. 
%Volgens de \hyperref[def:hoofdstelling_algebra]{Hoofdstelling van de Algebra} heeft de vergelijking $z^n=a+bi$ steeds $n$ oplossingen. 
% Het oplossen van de vergelijking $z^n=a+bi$ komt neer op het zoeken van de n-de machtswortels van $a+bi$.
 
\begin{example} Bereken alle vierkantswortels van $1-i$.
    \begin{hint}
    Je zoekt dus alle oplossingen van $z^2 = 1 - i$, met $z \in \C$.
    \end{hint}
    \begin{oplossing}
        We schrijven eerst beide leden van de opgave in de goniometrische schrijfwijze.
         
        De onbekende $z$ schrijven we als $z = r(\cos\theta+i\sin\theta)$. Het linkerlid van de vergelijking wordt dan volgens de formule van De Moivre $$z^2 = r^2 (\cos 2\theta+i\sin 2\theta).$$
         
        Voor het rechterlid $1-i$ geldt
        \[ |1-i| = \sqrt{2}, \qquad \arg (1-i) = - \frac{\pi}{4}, \]
        zodat
        \[ 1-i = \sqrt{2}( \cos(-\frac{\pi}{4}) + i \sin(-\frac{\pi}{4})). \]
         
        De
        vergelijking $z^2 = 1-i$ in de onbekende $z$ wordt zo een vergelijking in de onbekenden $r$ en $\theta$:
        $$r^2 (\cos 2\theta+i\sin 2\theta) = \sqrt{2} ( \cos(-\frac{\pi}{4}) + i \sin(-\frac{\pi}{4})).$$
        Twee complexe getallen in goniometrische schrijfwijze zijn gelijk aan elkaar als ze dezelfde modulus hebben, en hun argument gelijk is op een veelvoud van $2 \pi$ na.
        Hieruit volgt:
        \[ r^2 = \sqrt{2} \quad \text{met} \quad r\in \Rplus \qquad \text{en} \qquad 2 \theta = - \frac{\pi}{4} + 2 k \pi \quad \text{met} \quad k\in \Z\]
        Dus $r= 2^{1/4}$ en $\theta = - \frac{\pi}{8} + k \pi$, $k \in \Z$. We hebben de onbekenden $r$ en $\theta$ dus gevonden.
         
         Voor $k=0$ is
        $\theta = - \frac{\pi}{8}$ en voor $k=1$ is $\theta = -
        \frac{\pi}{8} + \pi = \frac{7\pi}{8}$. Voor alle andere waarden van $k$ krijgen we één van beide hoeken op een veelvoud van $2 \pi$ na. 
        Er zijn dus precies twee verschillende oplossingen
        \[ 
            z_1 = 2^{1/4}\left( \cos\left(-\frac{\pi}{8}\right) + i \sin\left(-\frac{\pi}{8}\right)\right)  \qquad \text{en} \qquad
            z_2 = 2^{1/4}\left( \cos\left(\frac{7\pi}{8}\right) + i \sin\left(\frac{7\pi}{8}\right)\right)  
        \]
         
    \end{oplossing}
\end{example}
 
\begin{example}
    Vind alle vierdemachtswortels uit $1$.
    \begin{oplossing}
    We zoeken alle oplossingen van de vergelijking $z^4=1.$
     
    We stellen $z = r(\cos\theta+i\sin\theta)$.
    Het linkerlid van de vergelijking wordt dan volgens de formule van De Moivre $$r^4(\cos 4\theta+i\sin 4\theta).$$
     
    Voor het rechterlid $1$ geldt
    \[ |1| = 1, \qquad \arg (1) = 0, \]
    zodat
    \[ 1 = 1(\cos 0+i\sin 0). \]
     
    De
    vergelijking $z^4=1$ in de onbekende $z$ wordt zo een vergelijking in de onbekenden $r$ en $\theta$:
     
    $$r^4(\cos 4\theta+i\sin 4\theta) =1(\cos 0+i\sin 0)$$
    Gelijkheid van complexe getallen leert ons dat $$\mbox{$r^4=1$ met $r\in \Rplus$ en
        $4\theta
        =0+2k\pi$ met $k\in \Z.$}$$ Dus $r=1$, want $r^4=1$ heeft slecht 1 reële oplossing, en $\theta=k\pi/2$ met $k\in
    \Z.$
    \\Dit levert vier verschillende wortels.
    \\$z_1=1(\cos 0+i\sin 0)=1$,
    \\$z_2=1(\cos \pi/2+i\sin \pi/2) =i$,
    \\$z_3=1 (\cos \pi+i\sin \pi) =-1,$
    \\$z_4=1 (\cos 3\pi/2+i\sin 3\pi/2)=-i.$
     
%   In het complex vlak liggen al deze wortels op de eenheidscirkel
%   rond de oorsprong, ze vormen de hoekpunten van een vierkant met
%   \'e\'en
%   hoekpunt in het punt $z=1$.
     
    Dit voorbeeld kan je veel sneller oplossen door $z^4-1$ te ontbinden in factoren:
    $$ z^4-1=(z^2-1)(z^2+1)$$
    De nulpunten hiervan zijn inderdaad 1,-1,i,-i. Ontbinden in factoren lukt echter niet meer bij bijvoorbeeld $z^5=1$, hiervoor moet je bovenstaande methode gebruiken.
    \end{oplossing}
    \end{example}
 
\begin{basicSkip}
    \begin{exercise} Bereken alle derdemachtswortels van $8i$.
    \begin{hint} Dit is dezelfde vraag als bereken alle oplossingen van $z^3=8i$, met $z\in \C$.\end{hint}
        \begin{oplossing}
            We schrijven eerst beide leden van de opgave in de goniometrische schrijfwijze.
             
            De onbekende $z$ schrijven we als $z = r(\cos\theta+i\sin\theta)$. Het linkerlid van de vergelijking wordt dan volgens de formule van De Moivre $$z^3 = r^3 (\cos 3\theta+i\sin 3\theta).$$
             
            Voor het rechterlid $8i$ geldt
            \[ |8i| = 8, \qquad \arg (8i) =  \frac{\pi}{2}, \]
            zodat
            \[ 8i = 8( \cos(\frac{\pi}{2}) + i \sin(\frac{\pi}{2})). \]
             
            De
            vergelijking $z^3=8i$ wordt:
            $$r^3 (\cos 3\theta+i\sin 3\theta) = 8( \cos(\frac{\pi}{2}) + i \sin(\frac{\pi}{2})).$$
            Twee complexe getallen in goniometrische schrijfwijze zijn gelijk aan elkaar als ze dezelfde modulus hebben, en hun argument gelijk is op een veelvoud van $2 \pi$ na.
            Hieruit volgt:
            \[ r^3 = 8 \quad \text{met} \quad r\in \Rplus \qquad \text{en} \qquad 3 \theta =  \frac{\pi}{2} + 2 k \pi \quad \text{met} \quad k\in \Z\]
            Dus $r= 2$, want $r^3=8$ heeft slecht 1 reële oplossing, en $\theta =  \frac{\pi}{6} + k \frac{2\pi}{3}$, $k \in \Z$.
             
            Voor $k=0$ is
            $\theta =  \frac{\pi}{6}$, voor $k=1$ is $\theta =
            \frac{\pi}{6} + \frac{2\pi}{3} = \frac{5\pi}{6}$ en voor $k=2$ is $\theta =
            \frac{\pi}{6} + \frac{4\pi}{3} = \frac{9\pi}{6} = \frac{3\pi}{2}$. Voor alle andere waarden van $k$ krijgen we één van deze hoeken op een veelvoud van $2 \pi$ na. Er zijn dus drie verschillende oplossingen
            \\$ z_1 = 2( \cos(\frac{\pi}{6}) + i \sin(\frac{\pi}{6}))   = 2(\frac{\sqrt3}{2} + i \frac12)= \sqrt 3 + i$,
            \\$ z_2 = 2( \cos(\frac{5\pi}{6}) + i \sin(\frac{5\pi}{6})) =2(- \frac{\sqrt3}{2} + i \frac12)=-\sqrt 3 +i $,
            \\$ z_3 = 2( \cos(\frac{3\pi}{2}) + i \sin(\frac{3\pi}{2})) =2(0 + i (-1))=-2i $.
        \end{oplossing}
        \end{exercise}
    \end{basicSkip}


Je kan de vergelijking $z^n=w$ oplossen voor een algemeen getal $w\in\C$, en dan krijg je formules die je daarna kan gebruiken. 
Het resultaat geeft volgende interessante eigenschap: de vergelijking $z^n=w$ heeft voor elke $w\in\Cnul$ precies $n$ oplossingen:

\begin{proposition}
    Voor elke $n\in\Nnul$ en $w=r(\cos\theta+i\sin\theta) \in\Cnul$ heeft de vergelijking $z^n = w$ precies $n$ verschillende complexe oplossingen
    $$
    z_k = \sqrt[n]{r}\left( \cos\left( \frac{\theta + 2\pi k}{n}\right) + i\sin\left( \frac{\theta + 2\pi k}{n}\right) \right), \text{ met } k\in\left\{0,1,2,\ldots,n-1\right\}
    $$
\end{proposition}
\begin{proof}
TODO ?
\end{proof}

Liefhebbers van de Euler notatie kunnen veel eenvoudiger schrijven
$$
    z_k = \sqrt[n]{r} e^\frac{\theta + 2\pi k}{n} \text{ met } k\in\left\{0,1,2,\ldots,n-1\right\}
$$



\end{document}