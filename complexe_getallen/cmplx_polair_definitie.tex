\documentclass{ximera}
\input{../preamble}
\addPrintStyle{..}
\begin{document}
    \author{Wiskunde Op Maat }
    \xmtitle{De polaire schrijfwijze van complexe getallen}{}
     
    \label{xim:complexe_getallen_polair}  % goniometrisch ???
 



In het vorig hoofdstuk werd een punt in het complexe vlak beschreven met behulp van carthesische coordinaten.  In dit hoofdstuk wordt een alternatieve manier behandeld om punten in het complexe vlak te beschrijven. Hiermee zullen verschillende operaties van complexe getallen, waaronder de vermenigvuldig, veel eenvoudiger en inzichtelijker worden. 

\begin{quickquestion*}{}
    Is er een andere manier waarmee je een punt in het vlak zou kunnen beschrijven? 
\end{quickquestion*}

 
De carthesische schrijfwijze associeert met een elk complex getal $z=a+bi$ het koppel \((a, b)\). Het reeël deel \(a\) komt overeen met de projectie op de reële as, het imaginair deel \(b\) komt overeen met de projectie op de imaginaire as. De polaire vorm associeert met elk complex getal in het vlak een koppel poolcoördinaten \((r, \theta)\).  Hierbij is $r=|z|$ de norm (afstand tot de oorsprong) en $\theta$ de hoek die de overeenkomstige vector maakt met de positieve reële as. Op die manier wordt elk complex getal beschreven met een uniek koppel poolcoordinaten. 


\begin{definition}
    De \textit{poolcoördinaten} van een complex getal zijn het koppel \((r, \theta)\), hierbij is $r=|z|$ de \textbf{modulus} en $\theta$ het \textbf{argument}. De hoek $\theta$ wordt gemeten in radialen, in tegenwijzerzin, en is slechts op een veelvoud van $2\pi$ na bepaald.

    \begin{image}%[\textwidth]
       \begin{tikzpicture}[scale=5]%,cap=round,transform canvas={scale=0.5}]
        
       \tikzmath{\hoek = 30; \myc = cos(\hoek); \mys = sin(\hoek);
           \hoekb = 20;}
        
       % Goniometrische cirkel
       %   \draw (0,0) circle (1cm);
       \draw[->] (-0.1,0) -- (1.3,0) node[above] {Re$(z)$};
       \draw[->] (0,-0.1) -- (0,0.9) node[below right] {Im$(z)$};
        
       \draw[color=blue,thick] (0:0)  -- node[right] {\small$r=|z|$} (\hoek:1);
       %
       \draw[color=black] (\hoek:1) node[name=P,circle, fill=black, radius=1pt,scale=0.8] {} node [yshift=1pt,above,align=center] {$z=a+bi=r(\cos\theta+i\sin\theta)$} ; 
       %
       \draw[dashed] ({cos(\hoek)},0) node[circle, fill=black, radius=1pt,scale=0.5] {} node[below] {\color{red}$a=r\cos\theta$} -- (P);
       \draw[dashed] (0,{sin(\hoek)}) node[circle, fill=black, radius=1pt,scale=0.5] {} node[left] {\color{red}$b=r\sin\theta$} -- (P);
       \draw[color=blue, ->] (0.3,0) arc (0:\hoek:0.3cm) node [midway,right] {$\theta$};  
        
       \end{tikzpicture}
    
    \end{image}

    Op die manier krijgt elk complexe getal een polaire schrijfwijze \(z = r(\cos \theta + i\sin\theta)\)

\end{definition}


 
 


 
 
%Indien $z=0$, stellen we per conventie $\theta=0+2k\pi, \; \mbox{met } k\in \Z$.
%\\ De hoek $\theta$ voor een gegeven $z$ is niet uniek bepaald.
%Wanneer we bij $\theta$ een geheel veelvoud van $2\pi$ bijtellen
%vinden we hetzelfde punt en dus hetzelfde complex getal terug.
 
% Er wordt niet gezegd wat r en theta zijn in deze definitie.
%\begin{definition}
%Een \textbf{goniometrische} (ook \textbf{polaire}) \textbf{voorstelling} van een complex getal $z = a+bi\in\C$ is
%$$
%\important{z = r(\cos\theta+i\sin\theta)}.
%$$
%waarbij de sinus en cosinus \textit{niet} worden uitgerekend.
%
%Hierbij is $r=|z|$ de \textbf{modulus} en $\theta$ het \textbf{argument} (ook \textbf{fase}).
%\end{definition}
 
 % Merk dus op dat aan de hand van de modulus $r$ en het argument $\theta$ van een complex getal $z$, het reële deel van $z$ gelijk is aan $a = r\cos \theta$, en het imaginaire deel gelijk is aan $b = r\sin \theta$.
 
%Men noteert dit soms ook als $r\angle \theta$.\\
 

 
 
Men kan een complexe getal in carthesische schrijfwijze $z=a+bi$ omvormen naar polaire vorm $z = r(\cos \theta + i\sin\theta)$ en vice versa. Het zal blijken dat sommige berekeningen veel eenvoudiger zijn in één van beide schrijfwijzes. De stelling van pythagoras en de goniometrische getallen leggen het verband tussen de carthesische en polaire schrijfwijze. 

\begin{proposition}[Transformatieformules cartesische en goniometrische  schrijfwijze]\label{eig:transformatie_complexe_getallen} \nl
     
    Een complex getal $z$ met goniometrische schrijfwijze $r(\cos \theta + i\sin\theta)$ heeft als cartesische schrijfwijze $a+bi$ met:
    \begin{center}
        \important{a =  r \cos \theta}\\
        \important{b =  r \sin \theta}
    \end{center}
    Een complex getal $z$ met cartesische schrijfwijze $a+bi$ heeft als goniometrische schrijfwijze $r(\cos \theta + i\sin\theta)$ met:
    \begin{align*}
    \important{r = \sqrt{a^2+b^2}} \\
    \important{\cos\theta  = \dfrac{a}{\sqrt{a^2+b^2}}}\\
    \important{\sin\theta  = \dfrac{b}{\sqrt{a^2+b^2}}}
    \end{align*}

\end{proposition}

 
\begin{quickquestion*}{}
    Gebruik de stelling van pythagoras en de defitie van de cosinus, sinus en tangens in een rechthoekige driehoek om bovenstaande formules af te leiden. 
\end{quickquestion*}
 

 
\begin{remark}\nl
     
    \begin{itemize}
        \item Omdat $\theta$ niet uniek bepaald is, heeft elk complex getal $z = r(\cos\theta+i\sin\theta)$ \textit{oneindig veel goniometrische schrijfwijzes} $z = r\left(\cos(\theta+k2\pi)+i\sin(\theta+k2\pi) \right)$, met $k\in\Z$. Vaak wordt er echter voor gekozen om $\theta$ tussen 0 en $2\pi$ te geven: we kunnen ons sneller voorstellen waar de hoek $\frac{5\pi}{4}$ ligt dan de hoek $\frac{1093\pi}{4}$.
        \item Twee complexe getallen in goniometrische schrijfwijze zijn gelijk aan elkaar als ze dezelfde modulus hebben, en hun argument gelijk is op een veelvoud van $2\pi$ na.
        %$$
        %r_1(\cos\theta_1+i\sin\theta_1)=r_2(\cos\theta_2+i\sin\theta_2)\quad\Leftrightarrow
        %\quad r_1=r_2 \; \mbox{ en }\; \exists k\in \Z:\theta_1=\theta_2+2k\pi.
        %$$
    \end{itemize}
     
\end{remark}



\begin{exercise}\nl
    Geef de goniometrische schrijfwijze van volgende complexe getallen waarvan het argument gemakkelijk grafisch gevonden kan worden:
    \begin{question} $z=1 $
        \begin{oplossing}
          Voor $z=1 $ is de modulus $1$ en het argument $0$, dus de goniometrische schrijfwijze is $$z= \cos 0 + i\sin 0 \text{  of  } z= \cos 2\pi + i\sin 2\pi.$$
        \end{oplossing}
    \end{question}
\begin{question} $z=i $
    \begin{oplossing}
     Voor $z=i $ is de modulus $1$ en het argument $\frac{\pi}{2}$, dus de goniometrische schrijfwijze is $$z= \cos\frac{\pi}{2} + i \sin \frac{\pi}{2}.$$  
    \end{oplossing}
\end{question}
    \begin{question} $z=1+i$
        \begin{oplossing}
            Voor $z=1+i$ is de modulus $\sqrt{2}$ en het argument $\frac{\pi}{4}$, dus de goniometrische schrijfwijze is $$z= \sqrt{2}(\cos \frac{\pi}{4}+i\sin\frac{\pi}{4}).$$ 
        \end{oplossing}
    \end{question}
    \begin{question} $z=\frac{\sqrt2}{2}+i\frac{\sqrt2}{2}$
        \begin{oplossing}
            Voor $z=\frac{\sqrt2}{2}+i\frac{\sqrt2}{2}$ is de modulus $1$ en het argument $\frac{\pi}{4}$, dus de goniometrische schrijfwijze is $$z = \cos \frac{\pi}{4}+i\sin\frac{\pi}{4}.$$
        \end{oplossing}
    \end{question}
     
    Merk op: bij het opschrijven van de goniometrische schrijfwijze met concrete getallen moet je de $\sin$ en $\cos$ laten staan. Als je die toch zou uitrekenen, krijg je immers terug de cartesische schrijfwijze.
\end{exercise}
 


 
% \begin{basicSkip}
% Je kan verder experimenteren met volgende applet: Je kan $z= a+bi=r(\cos \theta+ i  \sin \theta)$ verplaatsen en telkens $a=\text{Re}(z)$, $b=\text{Im}(z)$, $r=|z|$ en $\theta$ aflezen.
 
% \geogebra{NwzgUCQG}{887}{528}
% \end{basicSkip}
 
 

 

 
 
\begin{remark}\nl
     
    Als $a \neq 0$ kunnen we de tweede vergelijking delen door de eerste (waardoor de vierkantswortel wegvalt) en krijgen we
    $$
    \tan \theta = \frac{b}{a}  \quad \text{ als } a \neq 0
    $$
    Als $a >0$ dan is $\theta$ een hoek in het 1ste of het 4de kwadrant en is dus per definitie van de \hyperref[xim:cyclometrische_functies]{boogtangens }
    $$
    \theta  =  \bgtan \left(\frac{b}{a}\right) \quad \text{ als }  a>0
    $$
    Als $a<0$ dan is $\theta$ een hoek in het 2de of het 3de kwadrant en dan geldt
    $$
    \theta  =  \bgtan \left(\frac{b}{a}\right) + \pi \quad \text{ als }  a<0
    $$
    Als $a=0$, dan ligt $z=a+bi$ op de $y$-as en geldt
    \begin{align*}
    \theta & =  \frac{\pi}{2}  \quad\text{ als } a=0, b > 0\\
    \theta & = -\frac{\pi}{2} \quad\text{ als } a=0, b < 0
    \end{align*}
     
    Samengevat geeft dit voor $\theta$:
    $$
    \theta = \begin{cases}
    \bgtan \frac{b}{a}       & \text{ als } a>0        \\
    \pi + \bgtan \frac{b}{a} & \text{ als } a<0        \\   % met \pi+... lopen de breuken elkaar minder in de weg dan met ...+pi!
    \frac{\pi}{2}            & \text{ als } a=0, b > 0 \\
    -\frac{\pi}{2}           & \text{ als } a=0, b < 0
    \end{cases}
    $$


 
\end{remark}
% Bron: SPOC Complexe Getallen https://tube.geogebra.org/material/iframe/id/NwzgUCQG/width/887/height/528/border/888888/rc/false/ai/false/sdz/false/smb/false/stb/false/stbh/true/ld/false/sri/false/at/auto
 
 
 









\end{document}