\documentclass{ximera}
\input{../preamble}
\addPrintStyle{..}
\begin{document}
	\author{Wiskunde Op Maat}
	\xmtitle{Inleiding lineaire vergelijkingen}{}

   
  Het gelijkheidsteken '\( = \)' speelt een centrale rol in de wiskunde.  Het brengt de uitdrukkingen die links en rechts ervan staan op een heel spciale manier met elkaar in verband: ze zijn namelijk 'gelijk'. 


\begin{align*}
    \text{1 appel }                        & =  \text{1 appel                      }\\
    \text{2 appels }                       & =  \text{2 appels                    }\\
    \text{2 appels }                       & =  \text{1 appel } + \text{1 appel }    \\
    \text{1 groene appel}                  & =  \text{ 1 groene  appel      }\\
    \text{2 appels }                       & =  \text{ 1 groene appel + 1 groene appel}\\
    \text{2 appels }                       & =  \text{ 1 groene appel + 1 rode appel  }\\
    \text{2 appels } + \text{2 peren }     & =  \text{2 appels } + \text{2 peren}\\
    \text{2 appels } + \text{2 peren }     & =  \text{4 stukken fruit }\\
\end{align*}

Grote filosofische uitweidingen buiten beschouwing gelaten, hebben mensen bij bovenstaande gelijkheden een intuitief aanvoelen dat het gelijksheidsteken 
'\( = \)' aangeeft dat de 2 dingen op een manier 'gelijk' zijn. In de wiskunde en de wetenschappen zijn dergelijk gelijkheden alomtegenwoordig: 


\center{

\setlength{\extrarowheight}{2cm} 
\begin{tabular}{l  c r}
    \textbf{Pythagoras Theorema}    &  \( a^2 + b^2                                    = c^2              \)                            & \text{Pythagoras, 530BC}\\
    \textbf{Logaritmes }            &  \( \log(xy)                                     = \log(x) + log(y) \)                            & \text{John Napier, 1610}\\
    \textbf{Zwaartekracht }         &  \( F                                            = G \frac{m_1 m_2}{r^2} \)                       & \text{Newton, 1687}\\
    \textbf{Bayesiaanse statistiek }&  \( P(A|B)                                       = \frac{P(B|A) \cdot P(A)}{P(B)} \)                 & \text{Thomas Bayes, 1663}\\
    \textbf{The wortel uit -1}      &  \( i^2                                          = -1               \)                            & \text{Euler, 1750}\\
    \textbf{Relativiteit}           &  \( E                                            = mc^2      \)                                   & \text{Einstein, 1905}\\
    \textbf{Kwantummechanica}       &  \( \frac 1{c^2}{\partial^2 u \over\partial t^2} = \Delta u \)  & \text{Schrödinger, 1925}\\
    \textbf{Informatietheorie}      &  \( H(X)                                         = -\sum_{i=1}^{n} P(x_i) \log_b P(x_i)            \)  & \text{Claude Shannon, 1948}\\
    \textbf{Zwarte gaten}           &  \( S                                            = \frac{k A}{4 \hbar G}                           \)  & \text{Stephen W. Hawking, Jacob Bekenstein 1975}\\
\end{tabular}
}


Met deze pagina's kan je het oplossen van \textbf{lineaire vergelijkingen} inoefenen. Dit zijn uitdrukkingen van de vorm \(ax + b  = cx + d\).


Hierbij worden \textbf{termen} en \textbf{factoren} van kan gewisseld om te bepalen voor welke waarde van \( x\)  de gelijkheid geldig is. 

\textbf{Het is sterk aangeraden om hier veel op te oefenen}. In alle wetenschappen komen deze gelijkheden nog het hele middelbaar voor. 


\end{document}