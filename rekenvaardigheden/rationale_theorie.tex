\documentclass{ximera}
\input{../preamble}
\addPrintStyle{..}
\begin{document}
	\author{Wiskundeplan}
	\xmtitle{Rationale vormen}{}

Rationale vormen of gebroken lettervormen zijn ``breuken met letters''. Met de rekenregels voor \hyperref[xim:breuken_theorie]{breuken} en het \hyperref[xim:ontbinden_theorie]{ontbinden in factoren} kunnen deze uitdrukkingen vereenvoudigd worden. 
De belangrijkste rekentechnieken worden overlopen a.d.h.v enkele voorbeelden. 


\begin{example} Vereenvoudigen 

\begin{enumerate}
    \item \hyperref[xim:ontbinden_theorie]{Ontbind eerst de teller en de noemer in factoren.} 
    \item Deel vervolgens de teller en noemer door hun gemeenschappelijke factoren. 
 \end{enumerate}

    \begin{question} 
    \( \frac{a^2 b}{5ab^3} =\answer[onlineshowanswerbutton]{ \frac{a}{5b^2} } \)
    \begin{oplossing} \( \frac{a^2 b}{5ab^3} = \frac{a^{\cancel{2}} \cancel{b}}{5\cancel{a}b^{\cancel{3}}} = \frac{a}{5b^2}  \) \end{oplossing}
    \todo{bij het schrappen nog bijvoegen wat er overblijft. }
    \end{question}
    
    \begin{question} 
    \( \frac{a - b}{a^2 - b^2} =\answer[onlineshowanswerbutton]{\frac{1}{a + b}} \)
    \begin{oplossing} \( \frac{a - b}{a^2 - b^2} =\frac{a - b}{(a - b)(a + b)} = \frac{1}{a + b} \) \end{oplossing}
    \end{question}
    
    \begin{question} \( \frac{-x^3 + x^2 + 2x - 2}{x^2 - 1} =\answer[onlineshowanswerbutton]{ \frac{-x^2 + 2}{x + 1}} \) 
    \begin{oplossing} \( \frac{-x^3 + x^2 + 2x - 2}{x^2 - 1} = \frac{(x - 1)(-x^2 + 2)}{(x - 1)(x + 1)} = \frac{-x^2 + 2}{x + 1} \) \end{oplossing}   
    \end{question}
\end{example}


\begin{example} Optellen van twee rationale vormen 
\begin{enumerate}
    \item Ontbind de noemers in factoren.
    \item \hyperref[xim:breuken_theorie]{Zet de breuken op gelijke noemer}: dit is het kleinste gemeen veelvoud van de factoren in de noemers.
    \item Tel de tellers op en behoud de noemers. (De noemers blijven standaard ontbonden.)
    \item Vereenvoudig door gemeenschappelijke factoren te schrappen. 
\end{enumerate}


    
\begin{question}
    
    \( \frac{1}{a - b} + \frac{3}{b - a} =\answer[onlineshowanswerbutton]{\frac{-2}{a - b}} \)
    \begin{oplossing}
        \( \frac{1}{a - b} + \frac{3}{b - a}
        = \frac{1}{a - b} + \frac{-3}{a - b}
        = \frac{1 - 3}{a - b}
        = \frac{-2}{a - b}   \)      
    \end{oplossing}

\end{question}

\begin{question}
    
    \( \frac{-1}{x^2 + 2x} + \frac{2}{x^2 - 4} = \answer[onlineshowanswerbutton]{\frac{1}{x(x - 2)}} \)
            
\begin{oplossing}\nl
    \(
    \begin{array}{rcl}
        
        \frac{-1}{x^2 + 2x} + \frac{2}{x^2 - 4} 
        &=& \frac{-1}{x(x + 2)} + \frac{2}{(x - 2)(x + 2)}\\
        &=& \frac{-1}{x(x + 2)} + \frac{2(x - 2)}{(x - 2)(x + 2)} \\
        &=& \frac{-1}{x(x + 2)} + \frac{2x}{x(x - 2)(x + 2)}\\
        &=& \frac{-x + 2 + 2x}{x(x - 2)(x + 2)}\\
        &=& \frac{x + 2}{x(x - 2)(x + 2)}\\
        &=& \frac{1}{x(x - 2)}\\
    \end{array}
    \)           
\end{oplossing}         
\end{question}

\end{example}


\begin{example} Vermenigvuldigen van twee rationale vormen 
    \begin{enumerate}
        \item Vermenigvuldig de tellers met elkaar en de noemers met elkaar.
        \item Vergeet niet de uitkomst te vereenvoudigen.
    \end{enumerate}


    \begin{question}
        \( \frac{a^2 + 2ab + b^2}{a^2 - ab} \cdot \frac{3a^2}{2a + 2b} =\answer[onlineshowanswerbutton]{\frac{3a(a + b)}{2(a - b)}} \)        
        \begin{oplossing}\( \frac{a^2 + 2ab + b^2}{a^2 - ab} \cdot \frac{3a^2}{2a + 2b} = {\frac{(a^2 + 2ab + b^2) \cdot 3a^2}{(a^2 - ab) \cdot (2a + 2b)} = \frac{(a + b)^2 \cdot 3a^2}{a(a - b) \cdot 2(a + b)} = \frac{3a(a + b)}{2(a - b)}} \)  \end{oplossing}
    \end{question}
    
\end{example}





\begin{example} Delen van twee rationale vormen. 
\begin{enumerate}
    \item Maar gebruik van de rekenregel \( \frac{\frac{a}{b}}{\frac{c}{d}} = \frac{a}{b} \cdot \frac{d}{c} \). Dit mag enkel wanneer de teller en noemer allebei een breuk zijn. 
\end{enumerate}
    
\begin{question}
    \( \frac{a + 3}{1 + \frac{a}{a - 1}} =\answer[onlineshowanswerbutton]{\frac{(a + 3)(a - 1)}{2a - 1}} \)    

    \begin{oplossing} \( \frac{a + 3}{1 + \frac{a}{a - 1}} =\frac{a + 3}{\frac{a - 1 + a}{a - 1}} = \frac{a + 3}{\frac{2a - 1}{a - 1}} = (a + 3) \cdot \frac{a - 1}{2a - 1} = \frac{(a + 3)(a - 1)}{2a - 1} \) \end{oplossing}  
\end{question}
\end{example}



\end{document}