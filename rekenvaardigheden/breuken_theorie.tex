\documentclass{ximera}
\input{../../preamble}
\addPrintStyle{..}
\begin{document}
	\author{Wiskundeplan}
	\xmtitle{Rekenen met breuken}{}

Deze pagina geeft een overzicht van de rekenregels voor breuken. 
De focus ligt op rekenvaardigheid, voor een theoretisch meer nauwkeurige bespreken verwijzen we door naar een handboek. 


\begin{definition}\label{def:breuken}\nl
 
%Zij $a,b,c,d\in \R$.
Een \textbf{breuk} is een uitdrukking van de vorm $\frac ab$,  met $a,b\in\R$  en $b\neq0$.
\\[2mm]
We noemen $a$ de \textbf{teller} en $b$ de \textbf{noemer}.
Zodra de noemers verschillend zijn van $0$, geldt:
\begin{align*}
        \important{\frac{a}{b} = \frac{c}{d}}  &\iff \important{ad = bc}
             &\text{(gelijkheid van (getalwaarde van) breuken (kruisproduct))}
             %\label{def: gelijkwaardigheid breuken}
             \\[2mm]
        \frac{a}{b}+\frac{c}{d} \quad&\perdef\quad \frac{ad+cb}{bd}
             &\text{ (optelling (op gelijke noemer brengen))}
             %\label{def:optelling breuken}
             \\[2mm]
        \frac{a}{b} \cdot \frac{c}{d} \quad&\perdef\quad \frac{a\cdot c}{b\cdot d}
             &\text{(vermenigvuldiging (teller $\times$ teller, noemer $\times$ noemer)) }
             %\label{def: vermenigvuldiging breuken}
             \\[2mm]
        \frac{a}{b} : \frac{c}{d} \quad&\perdef\quad \frac ab \cdot \frac dc = \frac{a\cdot d}{b\cdot c}
             &\text{(deling (maal omgekeerde))}
             %\label {def: deling breuken}
\end{align*}

\end{definition}



Deze definities geven aanleiding tot een hele reeks rekenregels: 

\begin{proposition}\label{eig:rekenregels_breuken}\nl
 
{
\allowdisplaybreaks
\addtolength{\jot}{-2mm}  % hack: double // for html (??), but -2mm for pdf :( )
%\savebox\strutbox{$\vphantom{\dfrac{1^2}{1^2}^n}$}   % hack!
\begin{align*}
        \frac{a\cdot c}{a\cdot d} &=    \frac{\cancel{a}\cdot c}{\cancel{a}\cdot d} = \frac{c}{d}
            & \text{(gemeenschappelijke factor $a$ wegdelen)} \\
            \\
         \frac{c}{d}    &= \frac{a\cdot c}{a\cdot d}
            & \text{(met factor $a$ vermenigvuldigen)} \\
            \\
        a\cdot \frac{c}{d}   &= \frac{a\cdot c}{d}
            & \text{(getal maal breuk)}  \\
            \\
        \frac{a\cdot c}{d}   &= a\cdot \frac{c}{d}
            & \text{(factor uit teller halen)} \\
            \\
        \frac{a}{b}+\frac{c}{b}  &= \frac{a+c}{b}
            & \text{(breuken optellen, gelijke noemers)} \\
            \\
        \frac{a+c}{b} &= \frac{a}{b}+\frac{c}{b}
            & \text{(som in teller splitsen)}\\ % \hfill\text{(zelfde formule als vorige!)}\\
            \\
        \frac{\frac{a}{b}}{\ \frac{c}{d}\ } &= \frac ab \cdot \frac dc =  \frac{a\cdot d}{b\cdot c}
            & \text{(breuk gedeeld door breuk is maal omgekeerde)}\\
            \\
        \frac{\frac{a}{b}}{\ \frac{c}{b}\ } &= \frac ab \cdot \frac bc =  \frac{a}{c}
            & \text{(gelijke noemer $\frac 1b$  wegdelen)}\\
            \\
        \frac{\frac{a}{b}}{\ c\ } &= \frac ab \cdot \frac 1c = \frac{a}{b\cdot c}
            & \text{(breuk gedeeld door getal is maal in noemer)}\\
            \\
        \frac{a}{\ \frac{c}{d}\ } &= a\cdot \frac dc = \frac{a\cdot d}{c}
            & \text{(getal gedeeld door breuk is maal omgekeerde)}\\
            \\
\end{align*}
}
\end{proposition}


\begin{proposition}[Eenvoudige  gevolgen]\label{eig:rekenregels_breuken2}
    \[
        \begin{array}{rll}
        a + c &= d\cdot (\frac ad +\frac cd)
        & (\text{niet bestaande factor buiten haakjes brengen})\\
        a  &= \frac a1
        & (\text{van 'geen breuk' toch 'breuk' maken})\\
        a  &= \frac {1}{1/a}
        & (\text{van 'geen breuk' toch 'breuk' maken (variant)})\\
        a\cdot b  &= \frac {a}{1/b}
        & (\text{van een product een breuk maken})\\
        \frac ab  &= a\cdot \frac 1b
        & (\text{van een breuk een product maken})\\
        a\cdot b = 1  &\iff a = \frac 1b
        & (\text{van een product een breuk maken (variant)})\\
        \end{array}
    \]
\end{proposition}

herschrijven: 

je kan een niet bestaande factor buiten haakjes brengen 


Je kan van 'geen breuk' toch een breuk maken 

Je kan van een product een breuk maken 


Je kan van een breuk een product maken 







Bij het rekenen met breuken zijn er enkele foutieve operaties die voor leerlingen soms erg aanlokkelijk zijn. 
Doe jezelf (en je leerkracht...) een plezier en overtuig jezelf dat devolgende uitdrukkingen verkeerd zijn. 


\begin{remark}[Vaak gemaakte fouten]\nl \label{eig:niet-rekenregels_breuken}
     
    \textit{in het algemeen niet}:
    \\
    \\
    \[
    \begin{aligned}
        \frac{a}{b+c} &=    \frac{a}{b} + \frac{a}{c}\\
        \frac{a}{a+1} &=    \frac{\cancel{a}}{\cancel{a}+1}  = \frac{1}{1+1} = \frac{1}{2}\\
    \end{aligned}
    \]

\end{remark}

\end{document}