\documentclass{ximera}


\begin{document}
	\author{Wiskundeplan}
	\xmtitle{Ontbinden in factoren }{}
    \label{xim:ontbinden_theorie}


\textbf{Ontbinden in factoren} is het omzetten van een som (van termen) in een product (van factoren). Er bestaan hiervoor verschillende methodes. We ontbinden altijd zo ver mogelijk.


\begin{proposition} Afzonderen 

Voor het rekenen met lettervormen geldt:

\formulevb{ac+ad=a(c+d)}{4x + 4y = 4(x+y)} 

Distributiviteit toepassen in omgekeerde richting noemen we \textbf{afzonderen}.
\end{proposition}

\begin{example}       
    \begin{question} \( 24x^3 - 15x^2 + 3  =\answer[onlineshowanswerbutton]{ 3(8x^2 - 5x + 1) } \) \end{question}
    \begin{question} \( 24x^3 - 15x^2 + x  =\answer[onlineshowanswerbutton]{ x(24x^2 - 15x + 1) } \) \end{question}
    \begin{question} \( 24x^3 - 15x^2 + 3x  =\answer[onlineshowanswerbutton]{ 3x(8x^2 - 5x + 1) } \) \end{question}
    \begin{question} \( 24x^3 - 15x^2 + 3x^4  =\answer[onlineshowanswerbutton]{ 3x^2(x^2 + 8x - 5) } \) \end{question}
\end{example}
% \begin{exercise}
%     \begin{question} \( 6x^5 + 13x^2 + 2x^2 =\answer{ x^2(6x^3 +13x + 2)} \) \end{question}
%     \begin{question} \( 10x^2 + 4x + 2      =\answer{ 2(5x^2 +2x +1)    } \) \end{question}
% \end{exercise}

Soms kan je door haakjes toe te voegen bij een deel van de termen een factor afzonderen.
Dit speciale geval van afzonderen noemt men 'samennemen 2 aan 2'. 
Zoals volgend voorbeeld illustreert is het mogelijk dat hierna verder vereenvoudigd kan worden.

\begin{example} Samennemen 2 aan 2 
\[ 
\begin{array}{rclr}
    3x^3 - 6x^2 + x - 2 &=& (3x^3 - 6x^2) + (x - 2)\\
    &=& 3x^2(x - 2) + (x - 2)\\
    &=& (x - 2)(3x^2 + 1)
\end{array}  
\]

Merk op dat het ook mogelijk is om commutativiteit te gebruiken en de haakjes anders te plaatsen: 

\[ 
\begin{array}{rclr}
    3x^3 - 6x^2 + x - 2 &=& (- 6x^2 -2) + (3x^2 + x)\\
    &=& -2(3x^2 + 1) + x(3x^2 + 1)\\
    &=& (3x^2 + 1)(x - 2)
\end{array}  
\]
\end{example}




De distributiviteitseigenschap toepassen levert enkele interessante gelijkheden. Aangezien ze vaak voorkomen kregen ze een eigen naam: de merkwaardige producten. 

\begin{proposition} Merkwaardige producten 

Voor het rekenen met lettervormen:    
    \[
    \begin{array}{rclr}
        (a+b)^2    &=& a^2+2ab+b^2         &\text{(kwadraat van een som)}\\
        (a-b)^2    &=& a^2-2ab+b^2         &\text{(kwadraat van verschil via $(a+(-b))^2$)}\\
        a^2-b^2  &=&  (a+b)(a-b)  &      \text{(verschil van twee kwadraten)}\\[1cm]
        (a+b)^3 &=& a^3+3a^2b+3ab^2+b^3 &\text{(derdemacht van een som)}\\
        (a-b)^3 &=& a^3-3a^2b+3ab^2-b^3 &\text{(derdemacht van een verschil)}\\
    \end{array}
    \]
    
\end{proposition}


\begin{expandable}{proof}{Bewijs}
    De eigenschappen volgen rechtstreeks uit de distributiviteitseigenschap en de definitie van machtsverheffing. 

    \begin{itemize}
        \item \( (a+b)^2 = (a+b)(a+b) = a^2 + ab + ba + b^2 = a^2 + ab + ab + b^2 = a^2 + 2ab + b^2 \)
        \item \( (a-b)^2 = (a-b)(a-b) = a^2 - ab - ba + b^2 = a^2 - ab - ab + b^2 = a^2 - 2ab + b^2 \)
        \item \( (a+b)(a-b) = a^2 -ab +ba - b^2 =  a^2 -ab +ab - b^2 = a^2 - b^2 \)
        \item \( (a+b)^3 = (a+b)(a+b)^2 =(a+b)(a^2 + 2ab + b^2) = (a^3 + 2a^{2}b + ab^2)+(ba^2 + 2ab^2 + b^3) = a^3+3a^2b+3ab^2+b^3 \)
        \item \( (a-b)^3 = (a-b)(a-b)^2 =(a-b)(a^2 - 2ab + b^2) = (a^3 - 2a^{2}b + ab^2)-(-ba^2 + 2ab^2 - b^3) = a^3-3a^2b+3ab^2-b^3 \)
    \end{itemize}
\end{expandable}
    
Onderstaande kennisclip werkt zonder overbodige attributen de merkwaardige producten uit.  
\youtube{jyQTed3gHDA}
% \pdfOnly{\qrcode{https://youtu.be/jyQTed3gHDA}}

\begin{example}
    \begin{question} \( 4x^2 + 12xy + 9y^2   =\answer[onlineshowanswerbutton]{(2x + 3y)^2                   } \)\end{question}
    \begin{question} \( (3-2x)^2             =\answer[onlineshowanswerbutton]{9 - 12x + 4x^2              } \)\end{question}
    \begin{question} \( 25x^2 - 3            =\answer[onlineshowanswerbutton]{(5x - \sqrt{3})(5x + \sqrt{3})} \)\end{question}
    \begin{question} \( t^3 - 6t^2 + 12t - 8 =\answer[onlineshowanswerbutton]{(t - 2)^3                     } \)\end{question}
\end{example}


\begin{remark} Een merkwaardige visualisering

Het merkwaardige product \((a+b)^2 = a^2+2ab+b^2\) kan visueel worden voorgesteld.

\geogebra{yjajmaxb}{1000}{800}
\end{remark}


% De uitleg van wiskundeplan 

% \begin{example}
%     \item \textbf{Afzonderen} \\
%     \( 24x^3 - 15x^2 + 3x = 3x(8x^2 - 5x + 1) \)
    
%     \item \textbf{Merkwaardige producten}
%     \begin{itemize}
%         \item \( a^2 - b^2 = (a - b)(a + b) \) \quad \( 25x^2 - 3 = (5x - \sqrt{3})(5x + \sqrt{3}) \)
%         \item \( a^2 \pm 2ab + b^2 = (a \pm b)^2 \) \quad \( 4x^2 + 12xy + 9y^2 = (2x + 3y)^2 \)
%         \item \( a^3 \pm 3a^2b + 3ab^2 \pm b^3 = (a \pm b)^3 \) \quad \( t^3 - 6t^2 + 12t - 8 = (t - 2)^3 \)
%     \end{itemize}
    
%     \item \textbf{Samennemen 2 aan 2} \\
%     \( 3x^3 - 6x^2 + x - 2 = (3x^3 - 6x^2) + (x - 2) \) \\
%     \( = 3x^2(x - 2) + (x - 2) \) \\
%     \( = (x - 2)(3x^2 + 1) \)
% \end{example}

\end{document}