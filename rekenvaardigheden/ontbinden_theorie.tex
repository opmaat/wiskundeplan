\documentclass{ximera}
\input{../preamble}
\addPrintStyle{..}
\begin{document}
	\author{Wiskundeplan}
	\xmtitle{Ontbinden in factoren }{}


Ontbinden in factoren betekent: een som (van termen) omzetten in een product (van factoren). Er bestaan hiervoor verschillende methodes. We ontbinden altijd zo ver mogelijk.


\begin{proposition} Afzonderen

Voor het rekenen met letters en reële getallen geldt:
\[
    \begin{array}{rclr}
        ac+ad      &=&  a(c+d)   \\
    \end{array}
    \]
    
    Deze operatie is de 'omgekeerde' richting van de distributiviteits eigenschap. 
\end{proposition}

\begin{example}       
    \begin{question} \( 24x^3 - 15x^2 + 3x  =\answer[onlineshowanswerbutton]{ 3x(8x^2 - 5x + 1) } \) \end{question}
    \begin{question} \( 6x^5 + 13x^2 + 2x^2 =\answer[onlineshowanswerbutton]{ x^2(6x^3 +13x + 2)} \) \end{question}
    \begin{question} \( 10x^2 + 4x + 2      =\answer[onlineshowanswerbutton]{ 2(5x^2 +2x +1)    } \) \end{question}
\end{example}

Soms kan je door haakjes toe te voegen bij een deel van de termen een factor afzonderen.
Dit speciale geval van afzonderen noemt men 'samennemen 2 aan 2'. 
Zoals volgend voorbeeld illustreert is het mogelijk dat dit toelaat om daarna verder te vereenvoudigen:

\begin{example} Samennemen 2 aan 2 
\[ 
\begin{array}{rclr}
    3x^3 - 6x^2 + x - 2 &=& (3x^3 - 6x^2) + (x - 2)\\
    &=& 3x^2(x - 2) + (x - 2)\\
    &=& (x - 2)(3x^2 + 1)
\end{array}  
\]

Merk op dat het ook mogelijk is om commutativiteit te gebruiken en de haakjes anders te plaatsen: 

\[ 
\begin{array}{rclr}
    3x^3 - 6x^2 + x - 2 &=& (- 6x^2 -2) + (3x^2 + x)\\
    &=& -2(3x^2 + 1) + x(3x^2 + 1)\\
    &=& (3x^2 + 1)(x - 2)
\end{array}  
\]
\end{example}




De distributiviteitseigenschap toepassen levert enkele interessante gelijkheden. Aangezien ze vaak voorkomen kregen ze een eigen naam: de merkwaardige producten. 

\begin{proposition} Merkwaardige producten 

Voor het rekenen met letters en reële getallen geldt:    
    \[
    \begin{array}{rclr}
        (a+b)^2    &=& a^2+2ab+b^2         &\text{(kwadraat van een som)}\\
        (a-b)^2    &=& a^2-2ab+b^2         &\text{(kwadraat van verschil via $(a+(-b))^2$)}\\
        a^2-b^2  &=&  (a+b)(a-b)  &      \text{(verschil van twee kwadraten)}\\[1cm]
        (a+b)^3 &=& a^3+3a^2b+3ab^2+b^3 &\text{(derdemacht van een som)}\\
        (a-b)^3 &=& a^3-3a^2b+3ab^2-b^3 &\text{(derdemacht van een verschil)}\\
    \end{array}
    \]

    \begin{proof}
        Overtuig jezelf van de correctheid van deze rekenregels met hulp van de definities. 
    \end{proof}
    
\end{proposition}

\begin{example}
    \begin{question} \( 4x^2 + 12xy + 9y^2   =\answer[onlineshowanswerbutton]{(2x + 3y)^2                   } \)\end{question}
    \begin{question} \( (3-2x)^2             =\answer[onlineshowanswerbutton]{9 - 12x + 4x^2              } \)\end{question}
    \begin{question} \( 25x^2 - 3            =\answer[onlineshowanswerbutton]{(5x - \sqrt{3})(5x + \sqrt{3})} \)\end{question}
    \begin{question} \( t^3 - 6t^2 + 12t - 8 =\answer[onlineshowanswerbutton]{(t - 2)^3                     } \)\end{question}
\end{example}


\begin{remark} Een merkwaardige visualisering

Het merkwaardige product \((a+b)^2 = a^2+2ab+b^2\) kan visueel worden voorgesteld.

\geogebra{yjajmaxb}{950}{800}
\end{remark}


% De uitleg van wiskundeplan 

% \begin{example}
%     \item \textbf{Afzonderen} \\
%     \( 24x^3 - 15x^2 + 3x = 3x(8x^2 - 5x + 1) \)
    
%     \item \textbf{Merkwaardige producten}
%     \begin{itemize}
%         \item \( a^2 - b^2 = (a - b)(a + b) \) \quad \( 25x^2 - 3 = (5x - \sqrt{3})(5x + \sqrt{3}) \)
%         \item \( a^2 \pm 2ab + b^2 = (a \pm b)^2 \) \quad \( 4x^2 + 12xy + 9y^2 = (2x + 3y)^2 \)
%         \item \( a^3 \pm 3a^2b + 3ab^2 \pm b^3 = (a \pm b)^3 \) \quad \( t^3 - 6t^2 + 12t - 8 = (t - 2)^3 \)
%     \end{itemize}
    
%     \item \textbf{Samennemen 2 aan 2} \\
%     \( 3x^3 - 6x^2 + x - 2 = (3x^3 - 6x^2) + (x - 2) \) \\
%     \( = 3x^2(x - 2) + (x - 2) \) \\
%     \( = (x - 2)(3x^2 + 1) \)
% \end{example}

\end{document}