\documentclass{ximera}
\input{../../preamble}
\addPrintStyle{..}
\begin{document}
	\author{Wiskundeplan}
	\xmtitle{Oefeningen breuken}{}

 
\begin{exercise}   % FEB h3_rekenen_met_breuken
    Schrijf zo eenvoudig mogelijk.
    \begin{xmmulticols}[2]
        \begin{question} $\ds \frac{1}{6}+\frac{1}{2} =\answer{\frac{2}{3}}$
            \begin{oplossing} $\ds \frac{1}{6}+\frac{1}{2}= \frac{1}{6}+\frac{3}{6}= \frac{4}{6}= \frac{2}{3} $
            \end{oplossing}
        \end{question}
        \begin{question} $\ds \frac{2}{9}-\frac{1}{6} = \answer{\frac{1}{18}}$
            \begin{oplossing} $\ds \frac{2}{9}-\frac{1}{6}= \frac{4}{18}-\frac{3}{18}= \frac{1}{18}$
            \end{oplossing}
        \end{question}
        \begin{question} $\ds \frac{1}{3}-\frac{1}{9}+\frac{2}{27} = \answer{\frac{8}{27}}$
                \begin{oplossing} $\ds \frac{1}{3}-\frac{1}{9}+\frac{2}{27}= \frac{9}{27}-\frac{3}{27}+ \frac{2}{27}= \frac{8}{27}$ 
                \end{oplossing}
        \end{question}
        \begin{question} $\ds \frac{2}{3} \cdot \frac{5}{7} = \answer{\frac{10}{21}}$
            \begin{oplossing} $\ds \frac{2}{3} \cdot \frac{5}{7} =\frac{10}{21}$
            \end{oplossing}
        \end{question}
        \begin{question} $\ds \frac{15}{6} \cdot \frac{3}{2} = \answer{\frac{15}{4}}$
            \begin{oplossing} $\ds \frac{15}{6} \cdot \frac{3}{2} = \frac{5}{2} \cdot \frac{3}{2}= \frac{15}{4}$
            \end{oplossing}
        \end{question}
        \begin{question} $\ds \frac{2}{5} \cdot \frac{9}{22} \cdot \frac{4}{18} = \answer{\frac{2}{55}}$
           \begin{oplossing} $\ds \frac{2}{5} \cdot \frac{9}{22} \cdot \frac{4}{18} = \frac{\cancel{2}^1 \cdot \cancel{9}^1 \cdot 2}{5 \cdot \cancel{22}_{11} \cdot \cancel{9}_1} = \frac{2}{55}$
        \end{oplossing}
        \end{question}
        \begin{question} $\ds \frac{6}{5} : \frac{2}{15} = \answer{9}$
             \begin{oplossing} $\ds \frac{6}{5} : \frac{2}{15} = \frac{6}{5} \cdot \frac{15}{2}= \frac{\cancel{6}^3 \cdot \cancel{15}^3}{\cancel{5}_1\cdot \cancel{2}_1}= \frac{9}{1}=9$
             \end{oplossing}
        \end{question}
        \begin{question} $\ds \frac{12}{25} : \frac{18}{35} = \answer{\frac{14}{15}}$
            \begin{oplossing} $\ds \frac{12}{25} : \frac{18}{35} = \frac{12}{25} \cdot \frac{35}{18}=\frac{\cancel{12}^2\cdot \cancel{35}^7}{\cancel{25}_5 \cdot \cancel{18}_3} = \frac{14}{15}$
            \end{oplossing}
        \end{question}
        \begin{question} $\ds \frac{\frac{1}{4}}{\frac{3}{2}} = \answer{\frac{1}{6}}$
             \begin{oplossing} $\ds \frac{\frac{1}{4}}{\frac{3}{2}}= \frac{1}{4} : \frac{3}{2}= \frac{1}{4} \cdot \frac{2}{3} = \frac{1}{6}$
             \end{oplossing}
        \end{question}
        \begin{question} $\ds \frac{\frac{1}{2}+\frac{1}{4}}{\frac{1}{6}+\frac{1}{3}} = \answer{\frac{3}{2}}$
            \begin{oplossing} $\ds \frac{\frac{1}{2}+\frac{1}{4}}{\frac{1}{6}+\frac{1}{3}}= \frac{\frac{2+1}{4}}{\frac{1+2}{6}}=\frac{\frac{3}{4}}{\frac{3}{6}}= \frac{3}{4} \cdot \frac{6}{3}= \frac{3}{2} $
            \end{oplossing}
        \end{question}
    \end{xmmulticols}
\end{exercise}
 
 
\begin{exercise}
    Schrijf zo eenvoudig mogelijk.
    \begin{question} \( 20\cdot(\frac{5}{4}-\frac{4}{5})                                        =\answer[format=integer,onlineshowanswerbutton]{9} \) \end{question}
    \begin{question} \( -\frac{6}{27}+\frac{27}{1} + \frac{16+14}{9} -\frac{3}{14+13} - 3\cdot9 =\answer[format=integer,onlineshowanswerbutton]{3} \) \end{question}
    \begin{question} \( -\frac{(1-a)-2}{a+1}                                                    =\answer[format=integer,onlineshowanswerbutton]{1} \) \hfill($a\neq-1$)\end{question}
\end{exercise}
 
\begin{exercise}
Schrijf als een zo eenvoudig mogelijke breuk. Veronderstel dat alle uitdrukkingen bestaan.
    \begin{xmmulticols}[2]
\begin{question} \( \frac{a-b}{c}-\frac{a-2b}{2c            } =\answer[onlineshowanswerbutton]{\frac{a}{2c}        } \) \end{question}
\begin{question} \( \frac{\frac{a-b}{b}}{1-\frac{a}{b}      } =\answer[onlineshowanswerbutton]{-1                  } \) \end{question}
\begin{question} \( \frac{1-\frac{a+b}{b}}{\frac{a^{2}}{b}  } =\answer[onlineshowanswerbutton]{-\frac{1}{a}        } \) \end{question}
\begin{question} \( \ds \frac{a}{b}+\frac{b}{c              } =\answer[onlineshowanswerbutton]{\frac{ac+b^2}{bc}   } \) \end{question}
\begin{question} \( a+\cfrac{a}{1+a                         } =\answer[onlineshowanswerbutton]{\frac{2a+a^2}{1+a}  } \) \end{question}
\begin{question} \( 1+\cfrac{a}{1+a                         } =\answer[onlineshowanswerbutton]{\frac{1+2a}{1+a}    } \) \end{question}
\begin{question} \( 1+\cfrac{1}{1+a                         } =\answer[onlineshowanswerbutton]{\frac{2+a}{1+a}     } \) \end{question}
\begin{question} \( a+\cfrac{1}{1+a                         } =\answer[onlineshowanswerbutton]{\frac{1+a+a^2}{1+a} } \) \end{question}
\begin{question} \( \cfrac{1}{1+\cfrac{1}{1+a}              } =\answer[onlineshowanswerbutton]{\frac{1+a}{2+a}     } \) \end{question}
\begin{question} \( \cfrac{1}{1+\cfrac{1}{1+\cfrac{1}{1+a}} } =\answer[onlineshowanswerbutton]{\frac{2+a}{3+2a}    } \) \end{question}
    \end{xmmulticols}
\end{exercise}

\end{document}