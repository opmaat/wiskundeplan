\documentclass{ximera}
\input{../../preamble}
\addPrintStyle{..}
\begin{document}
	\author{Wiskundeplan}
	\xmtitle{Niveau 4}{}
	


	%waarom zijn deze rood? 


\begin{exercise} 
Pas het principe uit het voorbeeld toe om de veeltermen te ontbinden. 

	\begin{example}
		\begin{align*}
			x^4 + x^2 + 1 &= x^4 + 2x^2 + 1 - x^2 \\
						  &= (x^2 + 1)^2 - x^2 \\
						  &= (x^2 + x + 1)(x^2 - x + 1)
			\end{align*}
	\end{example}
    \begin{xmmulticols}
		\begin{question} \( x^4 + 1        =\answer[onlineshowanswerbutton]{(x^2 + \sqrt{2} x + 1)(x^2 - \sqrt{2} x + 1) } \) \end{question}
		\begin{question} \( x^4 + 3x^2 + 4 =\answer[onlineshowanswerbutton]{(x^2 + x + 2)(x^2 - x + 2)                   } \) \end{question}
		\begin{question} \( x^4 - 3x^2 + 4 =\answer[onlineshowanswerbutton]{(x^2 + \sqrt{7} x + 2)(x^2 - \sqrt{7} x + 2) } \) \end{question}
    
    \end{xmmulticols}    
\end{exercise}



% hoe zet ik hier het meerkueze 

\begin{exercise} VWO
	De som van de kwadraten van de reële oplossingen van \( x^{256} - 256^{32} = 0 \) is:
	\begin{itemize}
		\item (A) 8
		\item (B) 128
		\item (C) 512
		\item (D) 65536
		\item (E) \( 2 \cdot 256^{32} \)
	\end{itemize}  
\end{exercise}



%hier moet het antwoord ook komen. 

\begin{exercise} Finalevraag VWO 2012
	Stel \( n \) een natuurlijk getal. 
	Noem \( a \) het kleinste natuurlijk getal dat je van \( n \) moet aftrekken om een volkomen kwadraat te verkrijgen. 
	Noem \( b \) het kleinste natuurlijk getal dat je bij \( n \) moet optellen om een volkomen kwadraat te verkrijgen. 
	Bewijs dat \( n - ab \) een volkomen kwadraat is.
 
\end{exercise}








 






\end{document}