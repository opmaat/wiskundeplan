\documentclass{ximera}
\input{../preamble}
\addPrintStyle{..}
\begin{document}
	\author{Wiskundeplan}
	\xmtitle{Rekenen met machten}{}

Deze pagina geeft een overzicht van de rekenregels voor machten. 
De focus ligt op rekenvaardigheid. Voor meer nauwkeurige en theoretische bespreking verwijzen we door naar een handboek. 


\handouttrue

\begin{definition}(Machtsverheffing) \label{def:machten_light}\label{def:machten met gehele exponent}\
    
        
    % De \textbf{$n$-de macht} van een getal $a$ is genoteerd $a^n$, bijvoorbeeld $3^4$. \\
    
    Voor een natuurlijk getal $n$ en een reëel getal $a$ is de \textbf{$n$-de macht} van $a$ het getal
    
    \formulevb{
        a^n \perdef \underbrace{a\cdot a\cdot\ldots\cdot a}_{n\text{ factoren}}
        }{
            2^3=2\times2\times2
            }
    Men noemt $a$ het \textbf{grondtal},  $n$ de \textbf{exponent} en $a^n$  een \textbf{macht}.
            
    Negatieve exponenten geven per definitie het omgekeerde van de macht met de positieve exponent:
    
    \formulevb{a^{-n}\perdef\frac{1}{a^n}}{2^{-3}=\frac{1}{2^3} = \frac{1}{2\times2\times2}}
    
    Als de exponent 0 is (en het grondtal is niet 0), geldt per definitie dat  \important{a^0 = 1}, als $a\neq0$.
    \\
    De uitdrukkingen $0^0$ en $0^{-n}$ zijn \textit{onbepaald} en hebben \textit{geen betekenis}.

\end{definition}

\begin{exercise} Welke formules zijn waar?

    \begin{xmmulticols}[3]
    \begin{question} 
        \choiceFalse $\left(-2\right)^2 = -4$
        \begin{feedback} $(-2)^2 = (-2)(-2) = 4$ \end{feedback}
    \end{question}
    \begin{question} 
        \choiceTrue $-2^2 = -4$
        \begin{feedback} $-2^2 = - 2\cdot 2 = - 4 $ \end{feedback}
    \end{question}
    \begin{question} 
        \choiceFalse $2^{-2} = -4$
        \begin{feedback} $2^{-2} = \frac{1}{2^2} = \frac{1}{4} = 0,25$ \end{feedback}
    \end{question}
    
    \begin{question} 
        \choiceTrue $\left(\frac{1}{2}\right)^2 = \frac{1}{4}$
        \begin{feedback} Een kwadraat kan dus kleiner zijn dan het grondtal.\end{feedback}
    \end{question}
    \begin{question} 
        \choiceTrue $\left(\frac{1}{2}\right)^{-2} = 4$
        \begin{feedback} $ \left(\frac{1}{2}\right)^{-2} = \left(\frac{2}{1}\right)^{2} = 4$ \end{feedback}
    \end{question}
    \begin{question} 
        \choiceFalse $\left(\frac{-1}{2}\right)^{-2} = -4$
        \begin{feedback} $\left(\frac{-1}{2}\right)^{-2} = (-2)^2 = 4$\end{feedback}
    \end{question}
    \end{xmmulticols}

\end{exercise}

\begin{definition}(Machtswortels) \label{def:wortels_light}\label{def:wortels}\nl
     
    Een reëel getal $b$ is \textbf{\textit{een} vierkantswortel} van een reëel getal $a$ als 
    
    \formulevb{a = b^2}{2 \text{ en } -2 \text{ zijn wortels van } 4 \text{ want } 2^2 = 4 \text{ en } (-2)^2 = 4}

    Voor positieve $a$ noemen we het unieke \textit{positieve} getal $b$ zodat $b^2=a$ \textbf{\textit{de} vierkantswortel}, en die noteren we als $\sqrt{a}$ of ook als '$a$ tot de macht een half':
 
    \formulevb{ \sqrt{a} \perdef a^\frac 12 = b \iff b^2 = a \text{ en } b\geq 0  } {\sqrt{25} \perdef 25^{\frac{1}{2}} = 5}
 
    $\sqrt{a} = a^\frac{1}{2}$ is dus het unieke \textit{positieve} getal met kwadraat gelijk aan $a$.
 
    Meer in het algemeen schrijven we ook $n$-de machtswortels als 'tot de macht $\tfrac1n$':
     
    \formulevb{ \sqrt[n]{a} \perdef a^\frac 1n = b  \iff b^n = a} {\sqrt[3]{8} \perdef 8^\frac {1}{3} = 2 \Twant 2^3=8}
 
\end{definition}

\begin{exercise} Welke formules zijn waar?

    \begin{xmmulticols}[3]
    \begin{question} 
        \choiceTrue $\sqrt{4} = 2$
        \begin{feedback} $2 ^2 = 4$ \end{feedback}
    \end{question}
    \begin{question} 
        \choiceTrue $-\sqrt{4} = -2$
        \begin{feedback} $2 ^2 = 4$ \end{feedback}
    \end{question}
    \begin{question} 
        \choiceFalse $\sqrt{-4} = -2$
        \begin{feedback} $(-2)^2 = 4 \neq -4. \sqrt{-4}$ bestaat niet (in $\R$). \end{feedback}
    \end{question}
    \begin{question} 
        \choiceTrue $\sqrt[3]{-8} = -2$
        \begin{feedback} $(-2)^3 = (2)(-2)(-2) = -8$ \end{feedback}
    \end{question}
    \begin{question} 
        \choiceFalse $\sqrt[3]{9} = 3$
        \begin{feedback} $3\cdot3\cdot3 = 9\cdot3 = 27 \neq 9$ \end{feedback}
    \end{question}
    \begin{question} 
        \choiceFalse $\sqrt[3]{27} = 3$
        \begin{feedback} $3\cdot3\cdot3 = 9\cdot3 = 27$ \end{feedback}
    \end{question}

\end{xmmulticols}

\end{exercise}


Uit deze definities volgen enkele belangrijke rekenregels: 

\begin{proposition}[Rekenregels machten] \label{eig:rekenregels machten}\nl
 
    \savebox\strutbox{$\vphantom{\dfrac11^n}$}   % hack; doesn't work in html
 
    \formulevb{(xy)^n  = x^n y^n \quad \text{(macht van product)} }{(3y)^2 =3y\cdot 3y =3\cdot 3 \cdot y\cdot y = 3^2 y^2=9y^2 }
    \formulevb{x^{m}x^{n}  = x^{m+n} \quad \text{(product met gelijk grondtal)}  }{x^{3}x^{2}  = (x\cdot x\cdot x)\cdot (x\cdot x) = x^{3+2} = x^{5}    }
    \formulevb{\left(x^{m}\right)^{n} =x^{m\cdot n}\quad \text{(macht van macht)}  }{\left(x^{3}\right)^{2}=x^3 \cdot x^3 = (x \cdot x\cdot x)\cdot (x\cdot x\cdot x)=x^{3\cdot2}=x^6}
    \formulevb{\left(\frac{x}{y}\right)^{n} = \frac{x^{n}}{y^{n}} \quad \text{(macht van breuk)}}{\left(\frac{x}{2}\right)^{3} =
        \left(\frac{x}{2}\right) \cdot \left(\frac{x}{2}\right) \cdot \left(\frac{x}{2}\right) =
        \frac{x\cdot x \cdot x}{2 \cdot 2 \cdot 2} = \frac{x^{3}}{2^{3}}=\frac{x^3}{8} }
    % \frac{x^{m}}{x^{n}}    & = x^{m-n} & &  &\text{mits de noemer niet 0 is} %\text{als } x\neq 0 %x\neq0
    % \\
    % \left(\frac{x}{y}\right)^{n} & = \frac{x^{n}}{y^{n}} && & \text{mits de noemer niet 0 is}%y\neq0
    % \\
 
 
%   \begin{align*}
%   (xy)^n      & = x^ny^n             & (xy)^3  & =(xy)(xy)(xy)  = x^3y^3                  & \text{(macht van product)}\\
%   x^{m}x^{n}  & = x^{m+n}            & x^2 x^3 & = (xx)(xxx)    = x^{2+3} = x^5           & \text{(product gelijk grondtal)} \\
%   \left(x^{m}\right)^{n} & = x^{mn}  & (x^2)^3 & = (xx)(xx)(xx) = x^{2\times3} = x^6      & \text{(macht van macht)} \\
% \end{align*}


\end{proposition}

% \( \formulevb{\frac{x^{m}}{x^{n}}= x^{m-n} \quad \text{(breuk met gelijk grondtal)}} \) 




Bij het rekenen met breuken zijn er enkele foutieve operaties die voor leerlingen soms erg aanlokkelijk zijn. 
Doe jezelf (en je leerkracht...) een plezier en overtuig jezelf dat de volgende uitdrukkingen verkeerd zijn. 

\begin{remark} In het algemeen geldt niet:
    
    \renewcommand\CancelColor{\color{red}}
    

$$ \sqrt{a+b} \xcancel{=} \sqrt{a} + \sqrt{b}$$ 
$$ x^n + x^m \xcancel{=} x^{n+m} $$      
    
\end{remark}

\end{document}