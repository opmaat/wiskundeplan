\documentclass{ximera}
\input{../preamble}
\addPrintStyle{..}
\begin{document}
	\author{Wiskundeplan}
	\xmtitle{Rekenen met machten}{}



\begin{definition}(Machtsverheffing) \label{def:machten_light}\label{def:machten met gehele exponent}\
    
        
    De \textbf{$n$-de macht} van een getal $a$ is genoteerd $a^n$, bijvoorbeeld $3^4$. \\
    We noemen $a$ het \textbf{grondtal} en $n$ de \textbf{exponent}.
    
    Als $n$ een geheel positief getal is, dan is
    
    \formulevb{
    a^n \perdef \underbrace{a\cdot a\cdot\ldots\cdot a}_{n\text{ factoren}}
    }{
    2^3=2\times2\times2
    }
            
    Als de exponent negatief is, schrijven we de macht in de noemer:
    
    \formulevb{a^{-n}\perdef\frac{1}{a^n}}{2^{-3}=\frac{1}{2^3} = \frac{1}{2\times2\times2}}
    
    Als de exponent 0 is (en het grondtal is niet 0), is de macht altijd $1$:
    
    De uitdrukkingen $0^0$ en $0^{-n}$ zijn \textit{onbepaald} en hebben \textit{geen betekenis}.
        
\end{definition}



\begin{proposition}[Rekenregels machten] \label{eig:rekenregels machten}\nl
 
    \savebox\strutbox{$\vphantom{\dfrac11^n}$}   % hack; doesn't work in html
 
    \formulevb{(xy)^n  = x^n y^n \hfill \text{(macht van product)} }{(3y)^2 =3y\cdot 3y =3\cdot 3 \cdot y\cdot y = 3^2 y^2=9y^2 }
    \formulevb{x^{m}x^{n}  = x^{m+n} \quad \text{(product met gelijk grondtal)}  }{x^{3}x^{2}  = (x\cdot x\cdot x)\cdot (x\cdot x) = x^{3+2} = x^{5}    }
    \formulevb{\left(x^{m}\right)^{n} =x^{m\cdot n}\quad \text{(macht van macht)}  }{\left(x^{3}\right)^{2}=x^3 \cdot x^3 = (x \cdot x\cdot x)\cdot (x\cdot x\cdot x)=x^{3\cdot2}=x^6}
    \formulevb{\left(\frac{x}{y}\right)^{n} = \frac{x^{n}}{y^{n}} \quad \text{(macht van breuk)}}{\left(\frac{x}{2}\right)^{3} =
        \left(\frac{x}{2}\right) \cdot \left(\frac{x}{2}\right) \cdot \left(\frac{x}{2}\right) =
        \frac{x\cdot x \cdot x}{2 \cdot 2 \cdot 2} = \frac{x^{3}}{2^{3}}=\frac{x^3}{8} }
    % \frac{x^{m}}{x^{n}}    & = x^{m-n} & &  &\text{mits de noemer niet 0 is} %\text{als } x\neq 0 %x\neq0
    % \\
    % \left(\frac{x}{y}\right)^{n} & = \frac{x^{n}}{y^{n}} && & \text{mits de noemer niet 0 is}%y\neq0
    % \\
 
 
%   \begin{align*}
%   (xy)^n      & = x^ny^n             & (xy)^3  & =(xy)(xy)(xy)  = x^3y^3                  & \text{(macht van product)}\\
%   x^{m}x^{n}  & = x^{m+n}            & x^2 x^3 & = (xx)(xxx)    = x^{2+3} = x^5           & \text{(product gelijk grondtal)} \\
%   \left(x^{m}\right)^{n} & = x^{mn}  & (x^2)^3 & = (xx)(xx)(xx) = x^{2\times3} = x^6      & \text{(macht van macht)} \\
% \end{align*}
\end{proposition}

% \( \formulevb{\frac{x^{m}}{x^{n}}= x^{m-n} \quad \text{(breuk met gelijk grondtal)}} \) 




\begin{definition}%[$n$-de machtswortels ]
    \label{def:machten met rationale exponent}
    %We noemen $a^{\frac{1}{n}}$ de \textit{$n$-de machtswortel }uit $a$,
    %of de $n$-de wortel uit $a$ en noteren
 
    We schrijven de vierkantswortel ook als 'tot de macht een half':
    % \[
 
    \formulevb{ \sqrt{a} \perdef a^\frac 12 } {\quad \sqrt{25} \perdef 25^{\frac{1}{2}} = 5}
 
    % \]
    $a^\frac {1}{2}$ is het (positieve) getal waarvan de 2e macht gelijk is aan $a$.
    Vb:  $9^\frac {1}{2} = 3$ want $3^2=9$.
 
    Meer in het algemeen schrijven we ook $n$-de machtswortels als 'tot de macht $\tfrac1n$:
     
    \formulevb{ \sqrt[n]{a} \perdef a^\frac 1n = b  \iff b^n = a} {\sqrt[3]{8} \perdef 8^\frac {1}{3} = 2 \Twant 2^3=8}
 
 
\end{definition}



\end{document}